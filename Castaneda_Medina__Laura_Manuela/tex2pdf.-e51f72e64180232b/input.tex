% Options for packages loaded elsewhere
\PassOptionsToPackage{unicode}{hyperref}
\PassOptionsToPackage{hyphens}{url}
%
\documentclass[
]{article}
\usepackage{amsmath,amssymb}
\usepackage{iftex}
\ifPDFTeX
  \usepackage[T1]{fontenc}
  \usepackage[utf8]{inputenc}
  \usepackage{textcomp} % provide euro and other symbols
\else % if luatex or xetex
  \usepackage{unicode-math} % this also loads fontspec
  \defaultfontfeatures{Scale=MatchLowercase}
  \defaultfontfeatures[\rmfamily]{Ligatures=TeX,Scale=1}
\fi
\usepackage{lmodern}
\ifPDFTeX\else
  % xetex/luatex font selection
\fi
% Use upquote if available, for straight quotes in verbatim environments
\IfFileExists{upquote.sty}{\usepackage{upquote}}{}
\IfFileExists{microtype.sty}{% use microtype if available
  \usepackage[]{microtype}
  \UseMicrotypeSet[protrusion]{basicmath} % disable protrusion for tt fonts
}{}
\makeatletter
\@ifundefined{KOMAClassName}{% if non-KOMA class
  \IfFileExists{parskip.sty}{%
    \usepackage{parskip}
  }{% else
    \setlength{\parindent}{0pt}
    \setlength{\parskip}{6pt plus 2pt minus 1pt}}
}{% if KOMA class
  \KOMAoptions{parskip=half}}
\makeatother
\usepackage{xcolor}
\setlength{\emergencystretch}{3em} % prevent overfull lines
\providecommand{\tightlist}{%
  \setlength{\itemsep}{0pt}\setlength{\parskip}{0pt}}
\setcounter{secnumdepth}{-\maxdimen} % remove section numbering
\ifLuaTeX
  \usepackage{selnolig}  % disable illegal ligatures
\fi
\usepackage{bookmark}
\IfFileExists{xurl.sty}{\usepackage{xurl}}{} % add URL line breaks if available
\urlstyle{same}
\hypersetup{
  hidelinks,
  pdfcreator={LaTeX via pandoc}}

\author{}
\date{}

\begin{document}

\section{INF224 - Travaux Pratiques}\label{inf224---travaux-pratiques}

\subsection{General information}\label{general-information}

\subsection{Questions C++}\label{questions-c}

\subsubsection{4e Etape: Photos et
videos}\label{e-etape-photos-et-videos}

\begin{enumerate}
\def\labelenumi{\arabic{enumi}.}
\item
  The methods described, which are to be declared in the base class and
  overridden in the subclasses without implementation in the base class,
  are known as \textbf{pure virtual functions}. These functions are
  declared by assigning 0 to them in the declaration within the base
  class, making the base class abstract. This means objects of the base
  class cannot be instantiated directly, only objects of its subclasses
  can be.

  To declare this method:

  \texttt{virtual\ void\ play()\ const\ =\ 0;}
\item
  The reason why objects of the base class can no longer be instantiate
  after adding a pure virtual function is because the class becomes
  abstract. An abstract class is intended to provide a common interface
  and shared functionality for its subclasses, but it is incomplete on
  its own because of the presence of one or more pure virtual functions.
  Thus, we can only instantiate objects of classes that provide concrete
  implementations for all pure virtual functions inherited from their
  base class.
\end{enumerate}

\subsubsection{5e Etape: Traitement uniforme (en utilisant le
polymorphisme)}\label{e-etape-traitement-uniforme-en-utilisant-le-polymorphisme}

\begin{enumerate}
\def\labelenumi{\arabic{enumi}.}
\item
  The characteristic property of object-oriented programming that allows
  for the uniform treatment of a list containing both photos and videos,
  without concern for their specific types, is \textbf{polymorphism}.
  Polymorphism enables objects of different classes to be treated
  through the same interface, primarily through the use of base class
  pointers or references to access derived class objects. This
  capability is crucial for implementing a uniform behavior in a
  collection of objects that might belong to different classes but are
  related by inheritance.
\item
  In the case of C++, to make use of polymorphism, it is specifically
  necessary to:

  \begin{itemize}
  \tightlist
  \item
    Declare methods in the base class as virtual, so they can be
    overridden in derived classes. This includes both the methods for
    displaying attributes and for “playing” the objects.
  \item
    Use pointers or references to base class types to refer to objects
    of the derived classes. This is because C++ needs to know at compile
    time the size of the objects it deals with, and using pointers or
    references allows the program to handle objects of different sizes
    (due to potentially different data members in derived classes).
  \end{itemize}
\item
  The elements of the array in \texttt{main.cpp} should therefore be
  pointers (or smart pointers, like \texttt{std::unique\_ptr} or
  \texttt{std::shared\_ptr}) to the base class, not objects of the base
  class. This allows the array to hold references to both Photo and
  Video objects, treating them through their common base interface but
  allowing for the specific behavior of each derived class to be
  executed.

  \begin{itemize}
  \item
    This is different from Java, where all non-primitive types are
    inherently handled via references, and you can directly store
    objects of derived classes in an array or collection of the base
    class type without the explicit need for pointers.
  \item
    The reason for using pointers in C++ instead of objects is related
    to the slicing problem, where if objects of derived classes were
    assigned directly to elements of a base class array, any data or
    functions specific to the derived classes would be “sliced” off,
    leaving only the base class portion of the objects. Using pointers
    prevents slicing by allowing the program to access the full derived
    class objects.
  \end{itemize}
\end{enumerate}

\subsubsection{7e étape. Destruction et copie des
objets}\label{e-uxe9tape.-destruction-et-copie-des-objets}

\begin{enumerate}
\def\labelenumi{\arabic{enumi}.}
\item
  \textbf{Which classes need modifications to prevent memory leaks when
  objects are destroyed?}

  \begin{itemize}
  \tightlist
  \item
    Any class that allocates dynamic memory with new must release that
    memory with delete to prevent memory leaks. The Film class allocates
    memory for an array of chapter durations, so it needs a destructor
    to deallocate that memory.
  \end{itemize}
\item
  \textbf{What is the problem with copying objects that have instance
  variables which are pointers, and what are the solutions?}

  \begin{itemize}
  \item
    The problem with copying such objects is the shallow copy behavior
    of the default copy constructor and assignment operator provided by
    C++. A shallow copy duplicates the pointer values but not the
    pointed-to data, leading to situations where multiple objects point
    to the same memory location. This can cause undefined behavior when
    one object modifies the data or deallocates the memory, affecting
    all objects sharing that pointer.
  \item
    \textbf{Solutions include:}

    \begin{itemize}
    \tightlist
    \item
      Implementing a deep copy, where not only the pointers but also the
      pointed-to data are duplicated. This ensures each object manages
      its own copy of the data.
    \item
      Using smart pointers (e.g., \texttt{std::unique\_ptr},
      \texttt{std::shared\_ptr}) that automatically manage memory and
      can handle copying semantics correctly, though std::unique\_ptr
      requires explicit handling for copy operations since it owns the
      resource exclusively.
    \end{itemize}
  \end{itemize}
\end{enumerate}

\subsubsection{8e étape. Créer des
groupes}\label{e-uxe9tape.-cruxe9er-des-groupes}

\begin{enumerate}
\def\labelenumi{\arabic{enumi}.}
\item
  \textbf{Why use a list of object pointers?}

  \begin{itemize}
  \tightlist
  \item
    Using a list of pointers (specifically, pointers to the base class)
    allows the list to store objects of any derived class, enabling
    polymorphic behavior. This is necessary because the actual objects
    might have different sizes and behaviors, and we want to treat them
    through a common interface defined by the base class. This approach
    is similar to Java’s use of references for all objects, where Java
    automatically handles objects polymorphically without needing to
    explicitly use pointers.
  \end{itemize}
\item
  \textbf{Why objects are not destroyed when a group is destroyed?}

  \begin{itemize}
  \tightlist
  \item
    Since an object can belong to multiple groups, destroying an object
    when a group is destroyed would lead to undefined behavior when
    other groups try to access the destroyed object. This is managed by
    ensuring ownership of the objects’ lifetimes is handled outside the
    group, typically by the code that instantiates the objects. This
    approach requires careful memory management to prevent memory leaks,
    as you must ensure objects are deleted when no longer needed.
  \end{itemize}
\end{enumerate}

\subsubsection{10e étape. Gestion cohérente des
données}\label{e-uxe9tape.-gestion-cohuxe9rente-des-donnuxe9es}

\begin{enumerate}
\def\labelenumi{\arabic{enumi}.}
\item
  \textbf{How to ensure objects are only created through the
  MediaManager class to maintain database integrity?}

  \begin{itemize}
  \tightlist
  \item
    Make the constructors of the multimedia and group classes private or
    protected, ensuring they cannot be directly instantiated with new by
    external code. Then, make MediaManager a friend class of these
    classes, allowing MediaManager to access their constructors while
    other classes cannot.
  \end{itemize}
\end{enumerate}

\subsection{Questions Java}\label{questions-java}

\subsubsection{1ere Etape: Fenêtre principale et quelques
interacteurs}\label{ere-etape-fenuxeatre-principale-et-quelques-interacteurs}

\begin{enumerate}
\def\labelenumi{\arabic{enumi}.}
\item
  When we launch the program and interact with it by clicking the
  buttons to add text and resizing the window:

  \begin{enumerate}
  \def\labelenumii{\arabic{enumii}.}
  \item
    \textbf{Text Addition:} Each button press adds a specific line of
    text to the \texttt{JTextArea}, as programmed. The two buttons add
    their respective text lines, demonstrating how button actions can
    manipulate other components in the UI.
  \item
    \textbf{Window Resizing and Text Area Behavior:} Without a
    \texttt{JScrollPane}, the \texttt{JTextArea} might not handle
    overflow of text gracefully. As we add more text than the visible
    area can show, we might not be able to see all the text by
    scrolling. This is because \texttt{JTextArea} does not automatically
    include scrolling capability.
  \item
    \textbf{The Need for JScrollPane:} To make the \texttt{JTextArea}
    fully usable, especially when the amount of text exceeds the visible
    area, incorporating it into a \texttt{JScrollPane} is necessary. The
    \texttt{JScrollPane} provides a scrollable view of the
    \texttt{JTextArea}, allowing to navigate through all the entered
    text regardless of the amount, addressing usability concerns that
    become apparent through interaction and window resizing.
  \end{enumerate}
\end{enumerate}

\end{document}
